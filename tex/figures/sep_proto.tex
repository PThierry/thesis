%%
%%
%% sep_proto.tex for  in /home/phil/Travail/Doctorat/these/tex
%%
%% Made by Philippe THIERRY
%% Login   <Philippe THIERRYreseau-libre.net>
%%
%% Started on  lun. 27 mai 2013 14:51:01 CEST Philippe THIERRY
%% Last update lun. 27 mai 2013 14:51:01 CEST Philippe THIERRY
%%

\begin{pdfpic}
% Generated with LaTeXDraw 2.0.8
% Mon May 27 14:50:29 CEST 2013
% \usepackage[usenames,dvipsnames]{pstricks}
% \usepackage{epsfig}
% \usepackage{pst-grad} % For gradients
% \usepackage{pst-plot} % For axes
\scalebox{1} % Change this value to rescale the drawing.
{
\begin{pspicture}(0,-3.8939064)(13.44,3.9139063)
\definecolor{color364b}{rgb}{0.4980392156862745,0.5647058823529412,0.7176470588235294}
\definecolor{color645b}{rgb}{0.8862745098039215,0.8862745098039215,0.8862745098039215}
\definecolor{color641b}{rgb}{0.8156862745098039,0.8274509803921568,0.8980392156862745}
\psframe[linewidth=0.04,dimen=outer,fillstyle=solid,fillcolor=color641b](13.22,3.3660936)(8.58,-2.1939063)
\psframe[linewidth=0.04,dimen=outer,fillstyle=solid,fillcolor=color645b](8.58,3.3660936)(3.96,-2.1939063)
\psframe[linewidth=0.03,dimen=outer,fillstyle=hlines,hatchwidth=0.027999999,hatchangle=0.0,hatchsep=0.72120005](3.62,1.7460938)(0.0,-1.9939063)
\psframe[linewidth=0.03,dimen=outer,fillstyle=hlines*,hatchwidth=0.027999999,hatchangle=0.0,hatchsep=0.72120005](6.32,0.98609376)(4.08,-2.0339062)
\rput(1.6446875,-1.6239063){physique}
\rput(5.0646877,-1.6039063){physique}
\rput(1.6617187,-0.9039062){liaison}
\rput(5.081719,-0.9039062){liaison}
\rput(1.6614063,-0.16390625){réseau}
\rput(5.101406,-0.22390625){réseau}
\rput(1.6814063,0.5960938){transport}
\rput(5.121406,0.61609375){transport}
\rput(1.7279687,1.3160938){application}
\psframe[linewidth=0.04,dimen=outer,fillstyle=solid,fillcolor=color364b](6.3,1.7460938)(4.06,0.9660938)
\psline[linewidth=0.04cm,linestyle=dashed,dash=0.16cm 0.16cm](6.32,2.5660937)(6.32,-2.6739063)
\psframe[linewidth=0.03,dimen=outer,fillstyle=hlines*,hatchwidth=0.027999999,hatchangle=0.0,hatchsep=0.72120005](13.12,0.9660938)(10.88,-2.0539062)
\rput(11.864688,-1.6239063){physique}
\rput(11.881719,-0.92390627){liaison}
\rput(11.901406,-0.24390624){réseau}
\rput(11.921406,0.5960938){transport}
\psline[linewidth=0.04cm,linestyle=dashed,dash=0.16cm 0.16cm](10.88,2.5660937)(10.88,-2.6739063)
\rput(8.531406,3.6960938){{\it séparation protocolaire}}
\rput(1.7059375,2.1760938){pile TCP/IP}
\psline[linewidth=0.04,arrowsize=0.05291667cm 4.0,arrowlength=1.4,arrowinset=0.4]{->}(7.8,0.94609374)(7.8,-2.7339063)(9.18,-2.7339063)(9.18,0.92609376)
\psframe[linewidth=0.04,dimen=outer,fillstyle=solid,fillcolor=color364b](8.54,1.7460938)(6.3,0.9660938)
\psframe[linewidth=0.04,dimen=outer,fillstyle=solid,fillcolor=color364b](10.88,1.7460938)(8.64,0.9660938)
\psframe[linewidth=0.04,dimen=outer,fillstyle=solid,fillcolor=color364b](13.12,1.7460938)(10.88,0.9660938)
\rput(6.2315626,3.0160937){OS télécom domaine A}
\psframe[linewidth=0.04,dimen=outer](13.24,-2.2939062)(3.96,-3.4139063)
\rput(8.604688,-3.0639062){micro-noyau de sécurité}
\rput(10.820625,2.9960938){OS télécom domaine B}
\psline[linewidth=0.04,arrowsize=0.05291667cm 4.0,arrowlength=1.4,arrowinset=0.4]{->}(3.16,-3.8739061)(4.3,-3.8739061)(4.3,-2.0339062)
\psline[linewidth=0.04,arrowsize=0.05291667cm 4.0,arrowlength=1.4,arrowinset=0.4]{<-}(13.42,-3.8739061)(12.7,-3.8539062)(12.7,-2.0139062)
\end{pspicture} 
}
\end{pdfpic}
