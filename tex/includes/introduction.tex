%%
%%
%% introduction.tex for doctorat in /home/phil/Travail/Scolarité/Doctorat/doc/tex
%%
%% Made by Philippe THIERRY
%% Login   <Philippe THIERRYreseau-libre.net>
%%
%% Started on  mar. 24 nov. 2009 22:17:16 CET Philippe THIERRY
%% Last update lun. 07 déc. 2009 22:33:56 CET Philippe THIERRY
%%

\chapter{Introduction}
\minitoc

\section{Contexte de la thèse}
\paragraph{}
Ce sujet de thèse est choisi, en accord avec Thales Communication,
pour l'intérêt que je porte à la fois aux problématiques d'ordonnancement
temps réel et aux problématiques d'optimisation de traitement des flux réseau
sur les architectures multi-processeurs (comme par exemple dans les implémentations
fastpath).
\paragraph{}
Ainsi, le but de cette thèse est de travailler sur ces sujets afin d'éprouver les
solutions existantes, de mieux comprendre les différents éléments à prendre en
compte pour garantir le respect de contraintes temporelles associées aux flux
réseau et de mieux appréhender la possible adéquation entre l'architecture du noyau
Linux et les problématiques de flux temps réel, aujourd'hui gérées au
travers de mécanismes de classification implémentés dans TC.

\section{Problématique générale}
Aujourd'hui, les architectures multiprocesseurs permettent de fournir des
éléments de c\oe ur de réseau à forte capacité, gérant un grand nombre de flux aux
profils divers. De plus, les architectures de type multi-c\oe urs peuvent être un
bon compromis entre puissance et consommation, permettant ainsi de paralléliser
au mieux des traitements.

On distingue actuellement différentes architectures multic\oe ur:
SMP (Symetric MultiProcessor) ou AMP (Asymetric MultiProcessor).
L'architecture SMP utilise un noyau unique en mémoire, exécutée par l'ensemble des c\oe urs
alors que chaque processeur a son propre noyau dans une architecture AMP.\\
Certaines architectures permettent également l'emploi d'une configuration
mixtes, fournissant un environnement \textit{maître}, sur un sous ensemble de
processeurs en mode SMP, associés à une liste de processeurs AMP sur lesquels
s'exécutent de manière autonome des applicatifs en mode
\textit{single-exec}.\\
Ces architectures permettent ainsi par exemple l'implémentation d'algorithmes
de type \textit{fast-path}, dans le cadre du traîtement de flux
réseau.

\paragraph{}
La capacité d'optimiser au mieux l'utilisation de ces architectures pour exécuter une
application devrait permettre de satisfaire des contraintes applicatives de qualité de
service, exprimées sous la forme  de contraintes de latence de traversée d'un
flux, de contrainte de temps de réponse de bout en bout pour tout message d'un
flux, de contrainte de bande passante allouée à un flux ou encore le cloisonnement
des applications exécutés sur des processeurs différents.

\paragraph{}
La problématique de bon usage, au niveau logiciel, d'une architecture SMP, AMP
ou mixte est multiple. En effet, elle impacte à la fois la capacité du logiciel
à être réentrant, en réduisant au maximum les locks mémoire engendrés par une
exécution parallèle, mais également la manière dont ce dernier peut s'optimiser,
en allouant et désallouant de manière dynamique une charge CPU répartie, tout en
prenant en compte les impacts des phénomènes de cache.

\paragraph{}
Ainsi, la gestion de flux temps réels dans un routeur de c\oe ur de réseau fait partie
de ces éléments logiciels pouvant tirer grand parti de ces architectures, comme pouvant
être fortement handicapés par ces dernières en cas d'incompatibilités d'implémentation.
\paragraph{}
A ce jour, bien que des mécanismes de gestion de processus temps réel aient été spécifiés au
travers de normes internationales et d'algorithmes reconnus (comme la famille EDF - Earliest Deadline First) ou
encore au travers de véritable langages (comme l'implémentation fournie par IBM dans le cadre
de ses Mainframes avec JCL - Job Control Language), l'implémentation de ces derniers reste
très complexe.\\

L'arrivé massive de processeurs permettant de fournir une base matérielle SMP ou AMP
à coût et à consommation faible implique une réorganisation de l'architecture logicielle
de traitement, ainsi que des entités associées.
\paragraph{}
De plus, certaines optimisations d'ordonnancement de flux, comme l'emploi fréquent du
fastphath, restent souvent limitées à des implémentations propriétaires fermées.
\paragraph{}
Cependant, les modifications effectuées ces dernières années dans l'ordonnanceur du noyau
Linux afin de réduire les lock mémoires sur les architectures multi-c\oe urs, les
modifications de la chaîne de traitement réseau réduisant comme une peau de chagrin
les éléments non réentrant au niveau driver, bottom-halves et pile réseau sont
quelques exemples des grands travaux en cours de réalisation afin d'optimiser
au mieux le logiciel pour ces architectures et montrant l'intérêt croissant porté
par la communauté à ces problématiques.

\paragraph{}
Dans ce contexte, un système temps réel critique est typiquement un système exécutant
une application constituée de flux par exemple périodiques ayant des contraintes temporelles
strictes. Un flux périodique est caractérisé par sa durée d'exécution pire cas (WCET),
sa période et son échéance de remise au plus tard.\\
%
Pour de tels systèmes, il est nécessaire de garantir que toute demande de traitement
d'un flux sera toujours terminée avant une échéance de remise au plus tard bornée.
Cette garantie peut être obtenue par l'établissement de conditions de faisabilité (CF)
qui lorsqu'elles sont vérifiées garantissent que l'ensemble des flux respectent leurs
contraintes temporelles. Ces CF utilisent les paramètres des flux (WCET, période ou
échéance).\\
%
Les analyses de sensibilité issues de l'état de l'art sont utiles pour caractériser le
comportement d'un système temps réel en cas de déviation du système par
rapport à ses spécifications. Il peut alors être intéressant dans une phase de
dimensionnement de déterminer les limites acceptables de variation des WCETs,
périodes ou échéances permettant de garantir l'ordonnançabilité des flux. Un changement
de WCET peut par exemple survenir lorsque l'on cherche à optimiser la consommation d'énergie
du système temps réel. Réduire  la vitesse du processeur permet de moins consommer d'énergie
mais conduit à augmenter les WCETs des flux.\\
Il peut être intéressant en contexte multiprocesseur d'étudier dans quelle mesure un
ordonnancement multiprocesseur valide, reste valide lorsque l'on augmente ou diminue les
WCETs des flux. L'état de l'art sur l'ordonnancement temps réel montre que la période des
flux est un paramètre très sensible. Plusieurs cas d'anomalies d'ordonnancement ont été
identifiés : un jeu de tâches est ordonnançable pour certaines valeurs de périodes mais
augmenter la période de certaines tâches conduit à un problème d'ordonnancement non faisable.
\paragraph{}
L'objectif de cette thèse est d'étudier les problématiques d'ordonnancement de flux temps réel,
en contexte multiprocesseur (SMP ou AMP) pour les deux grandes familles d'ordonnancement
multiprocesseur que sont l'ordonnancement par partition (partitioned scheduling) et l'ordonnancement
global (global scheduling), mais aussi dans le cadre des nouvelles techniques
d'ordonnancement semi-partitionné (semi-partitioned scheduling) permettant la
migration des tâches selon un motif statique \cite{sempart}.\\
%
Les CF pour le partitionnement consistent à utiliser une heuristique de placement
des flux sur les processeurs et d'appliquer ensuite une CF pour déterminer si l'ensemble
des flux est bien ordonnançable. Les CF pour l'ordonnancement global permettent
de considérer qu'un flux soit exécuté successivement sur plusieurs processeurs. L'état de l'art montre
que l'ordonnancement global offre plus de perspectives en terme de faisabilité, il est
cependant plus complexe à analyser. Le problème de détermination du coût de migration
des processus et de gestion de la mémoire pour des architecture SMP ou AMP est encore
largement ouvert.
\paragraph{}

Une validation des résultats théoriques établis dans cette thèse sera réalisée dans
le cadre d'une plateforme basée sur un noyau Linux. Linux est un architecture noyau
extrèmement riche, fournissant des capacités de traitement réseau reconnues, sur
laquelle plusieurs expérimentation de noyau temps réel ont été proposées (RTAI, RT-Linux et LITMUS-RT).

L'implémentation de solution algorithmiques pour l'ordonnancement de flux, au travers
par exemple de l'architecture de TC et des mécanismes complexes sous-jacents dans
le chemin réseau, dont la dynamicité de mise à jour est constante, fournit un sujet
d'étude passionnant et sur lequel il serait bon de s'appuyer afin de prendre en
compte les problématiques SMP ou AMP et temps réel.


