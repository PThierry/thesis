%%
%%
%% problematique.tex for doctorat in /home/phil/Travail/Scolarité/Doctorat/doc/tex
%%
%% Made by Philippe THIERRY
%% Login   <Philippe THIERRYreseau-libre.net>
%%
%% Started on  mer. 25 nov. 2009 20:37:12 CET Philippe THIERRY
%% Last update Tue Mar 16 17:03:30 2010 Philippe THIERRY
%%


\chapter{PROBL�MATIQUE}
\doMinitoc

\section{Besoins en terme de flux temps r�els}

\subsection{Rappels}
\paragraph{}
Aujourd'hui, la plupart des r�seaux s'appuient sur des protocoles asynchrones de type
{\it best effort} comme \index{Protocole!IP}\useglosentry{*}IP.
Il existe des cependant des cas particuliers (r�seaux hertziens, r�seaux tr�s contraints,
r�seaux avec forte garantie de latence de bout en bout), o� il est n�cessaire de fournir
des solutions syst�mes permettant, tant au niveau protocolaire qu'au niveau noeud de
communication, de fournir des garanties fortes sur la travers�e des donn�es.

\paragraph{}
De plus, avec l'accroissement des bandes passantes, il devient imp�ratif de consid�rer
la probl�matique de temps de travers�e au niveau des routeurs.\\
La co�t du traitement logiciel qu'implique le routage d'un paquet au sein d'un noeud de
communication, impacte directement ces performances en terme de bande passante support�e,
et donc sa capacit� � s'int�grer en c{\oe}ur de r�seau.

\subsection{Les besoins en terme de latences garanties}

\section{Cas d'usage}

\subsection{Les r�seaux synchrones bas d�bit}

\subsection{Les architectures r�seau temps r�el}

\subsubsection{Les r�seaux AFDX}

