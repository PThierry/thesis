%%
%%
%% proposition_software.tex for thesis in /doctorat/these/tex
%%
%% Made by Philippe THIERRY
%% Login   <Philippe THIERRYreseau-libre.net>
%%
%% Started on  Thu Mar 18 17:20:20 2010 Philippe THIERRY
%% Last update Mon Jul 19 15:18:08 2010 Philippe THIERRY
%%

\chapter{D�finition d'une plateforme logicielle pr�dictible et s�curis�e}

\section{D�finition d'une s�curisation par la compartimentation}

\subsection{Int�r�ts du m�cansime de compartimentation}

\subsection{D�finition des compartiements de traitement n�cessaire au besoin}

\subsection{Exigences sur l'architecture logicielle}

\subsection{D�finition d'une politique d'ordonnancement optimis� pour les flux r�seaux}

\subsection{Prise en compte de l'impact des traitements noyaux sur l'ex�cution des taches temps r�elles}

\subsubsection{Int�gration d'une politique de s�curit� par partition}

\section{Maquettage sur une architecture logicielle cloisonnante bas�e sur la solution logicielle LXC}

% LXC
\subsection{D�finition d'une politique d'ordonnancement optimis� pour les flux r�seaux}

\subsubsection{Cas des flux r�seaux synchrones}

Utilisation de threads kernel g�r�s comme des taches temps r�elles

\subsubsection{Cas des flux r�seaux asynchrone}

Cas des protocoles CSMA/CD, CSMA/CA. Ordonnancement de type temps r�el 'souple'.

\subsection{Prise en compte de l'impact des traitements noyaux sur l'ex�cution des taches temps r�elles}

\subsection{Double ordonnancement et architectures � base de micro-noyaux}

\subsubsection{Int�gration d'une politique de s�curit� par partition}

% PikeOS
\subsection{Description d'une architecture logicielle bas�e sur un SSK}
