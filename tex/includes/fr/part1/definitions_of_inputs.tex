%%
%%
%% definitions_of_inputs.tex for thesis in /doctorat/these/tex
%%
%% Made by Philippe THIERRY
%% Login   <Philippe THIERRYreseau-libre.net>
%%
%% Started on  Fri Mar 12 16:23:03 2010 Philippe THIERRY
%% Last update Wed Mar 17 12:53:23 2010 Philippe THIERRY
%%

\chapter{D�FINITION DES DONN�ES D'ENTR�E}
\doMinitoc

Descriptions des �l�ments de base des probl�matiques temps r�el.

\section{WCET : premier �l�ment dimensionnant}

\paragraph{}
Le \gls{wcet} ({\it Worst Case Execution Time}) est la valeur mesur�e ou construite permettant de qualifier, pour
une tache donn�e, le temps maximum n�cessaire � son ex�cution.\\
Dans un environnement id�al, l'ex�cution d'une tache est garantie inf�rieure � cette valeur.

\subsection{D�finition du WCET}

\subsection{Mesure du WCET}

\section{Comportement des ensembles de taches}

\subsection{Les ensembles p�riodiques}

Les taches sont cr��es � une p�riode fixe.

\subsection{Les ensembles sporadiques}

Les taches sont cr��es avec une p�riode \textit{minimum} fixe, aucun maximum n'est d�finit.

\subsection{Les ensembles ap�riodiques}

La cr�ation de tache se fait de mani�re al�atoire dans le temps.

\subsection{Les ensembles fen�tr�s}

Une fen�tre de temps ${\Delta}t$ est d�finie, et on lui associe un nombre d'interruptions
maximum. Cette valeur est utilis�e en entr�e de la mesure du WCET.

\section{La dynamique des ensembles de taches}

\subsection{Les ensembles concrets}

\subsection{Les ensembles non concrets}

