%%
%%
%% software_security.tex for thesis in /doctorat/these/tex
%%
%% Made by Philippe THIERRY
%% Login   <Philippe THIERRYreseau-libre.net>
%%
%% Started on  Thu Jul 15 16:04:45 2010 Philippe THIERRY
%% Last update Mon Jul 19 15:14:28 2010 Philippe THIERRY
%%

\chapter{Principes g�n�raux de la s�curit� logicielle}

{\it
Pr�sentation rapide...
}

\doMinitoc

\section{D�finir une gestion s�curis�e de la m�moire}

\subsection{Garantir une s�gr�gation des acc�s m�moire des taches}

\paragraph{}

\subsection{� propos du d�ploiement d'un processus en m�moire}

\paragraph{}
Un applicatif, construit dans un fichier binaire lors de l'�dition de lien, poss�de un certain nombre de sections
servant de conteneur � des donn�es diff�rentes, et ayant en cons�quence des propri�t�s de s�curit� (en terme d'acc�s)
�galement diff�rente. De ce fait, lors du d�ploiement de ces diff�rentes section en m�moire vive, il est n�cessaire
de consid�rer pour chacune de ces section des propri�t�s d'acc�s sp�cifiques.

\paragraph{}
De plus, le positionnement de ces sections en m�moire ne doit pas �tre pr�dictible, sous peine de simplifier les
attaques par r��criture sur des sections fragiles comme la pile.

\section{De l'int�rer de cloisonner les taches}
\paragraph{}

\section{Les probl�matiques de s�curit� des noyaux}
\paragraph{}
