%%
%%
%% hardware_impacts.tex for thesis in /doctorat/these/tex
%%
%% Made by Philippe THIERRY
%% Login   <Philippe THIERRYreseau-libre.net>
%%
%% Started on  Fri Mar 12 16:36:41 2010 Philippe THIERRY
%% Last update Mon Aug 30 17:00:21 2010 Philippe THIERRY

\chapter{De la d�finition d'une architecture logicielle pour le MLS orient�e traitement de flux}
\doMinitoc

\section{D�finition des moniteurs de s�curit�}

\subsection{A propos d'une architecture de diode logicielle pour le transfert agnostique de donn�e}

Diode ESSK (citer brevet (� valider))

\subsection{De la probl�matique d'exploitation des canaux auxiliaires: � propos des profils de flux}

Dernier article: �metteur p�riodique

\subsection{De la probl�matique d'exploitation des canaux auxiliaires: les caches processeur}

Article sur les caches, probl�matique des flush de cache (� prendre en compte
dans l'ordonnancement

\section{Moniteurs de s�curit�, fonctions temps r�el et environnements non s�curisables}

\section{R�partition des moniteurs en environnement multi-processeur}

\subsection{Localisation des moniteurs et des compartiments par domaines de
s�curit�}

\chapter{Temps r�el, assurance de Qualit� de Service IP et flux � multiple criticit�s}

\section{Ordonnancement hierarchique et Time Division Multiplexing}

D�finition du probl�me, besoin de s�curit�

\section{Priorisation par criticit�, modification de la QoS pour la s�curit�}

Changement de criticit�: priorisation aux flux dont la QoS IP n'est pas la
plus �lev�e

