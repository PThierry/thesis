%%
%%
%% introduction.tex for doctorat in /home/phil/Travail/Scolarit�/Doctorat/doc/tex
%%
%% Made by Philippe THIERRY
%% Login   <Philippe THIERRYreseau-libre.net>
%%
%% Started on  mar. 24 nov. 2009 22:17:16 CET Philippe THIERRY
%% Last update Wed Apr  6 13:03:47 2011 Philippe THIERRY
%%

\chapter{Introduction}
\doMinitoc

\section{About the thesis context}
\paragraph{}
This thesis subject has been chosen with Thales Communications, for the common interest on the
scheduling problematics associated to the network management functions and the associated security
and partitioning in the case of multiple criticity network flows.

\paragraph{}
Thus, the aim of this thesis is to work on the formalisation of the software and hardware properties
permitting the execution of real-time network management functions on hybrid real-time and secure
system.\\
In order to be representative of the industry needs and to take into account the existing work which
have been done during these last years, it is necessary to consider the usage of existing operating
systems and hardware, in a perspective of defining a security and real-time efficient system in
order to compensate the fragility of such software environment, as much in term of real-time
properties as in term of security considerations.

\section{About the targeted hardware environment}

\paragraph{}
Today, multi-core architecutres permit a huge efficiency through the usage of fully parallel
executions. NoC-based architecture also permit a concurrent usage of the local ship network,
reducing the bus contenion and the border-effects of concurrent execution of multiple fonctions.\\
Such architectures are more and more used in embedded systems, like last ARM Cortex designs
supporting multiple cores.

\section{Introducing my work}

\paragraph{}
During my thesis, I am trying to define a functional system supporting multiple real-time
environments with both real-time and security guarantee between in order to support both real-time
treatments and security enanced partitioning.\\
Such software architecture permits the usage of a single hardware plateform in order to support
multiple needs, which seriously reduce the cost in term of both wheight and power consumption in
embedded environments.

\paragraph{}
Such architecture need to rely on various hardware properties in order to permit such guarantees.
It is then needed to formalize which properties should be respected in such system and validate the
compliance of existing hardware plateforms with such properties.

