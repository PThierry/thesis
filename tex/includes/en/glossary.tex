%%
%%
%% glossary.tex for thesis in /doctorat/these/tex
%%
%% Made by Philippe THIERRY
%% Login   <Philippe THIERRYreseau-libre.net>
%%
%% Started on  Tue Mar 16 16:23:32 2010 Philippe THIERRY
%% Last update lun. 04 avril 2011 19:18:09 CEST Philippe THIERRY
%%

%\usepackage[style=long,border=none,header=plain,cols=3,toc,number=section]{glossary}
\usepackage[style=long3colheaderborder]{glossaries}
\makeglossary

\newglossaryentry{ip}{name=IP,description={Internet Protocol}}
\newglossaryentry{atm}{name=ATM,description={Protocole de commutation de circuit, recouvrant les couches liaison, r�seau et transport.}}
\newglossaryentry{x25}{name=X25,description={\FIXME}}

\newglossaryentry{8021q}{name={802.1q},description={\FIXME}}
\newglossaryentry{diffserv}{name={DiffServ},description={Champs de l'en-t�te IP permettant de d�finir une diff�renciation de service en fonction de sa valeur}}
\newglossaryentry{qos}{name={QoS},description={{\it Quality Of Service}, ensemble de moyens informatique permettant de fournir une meilleure garantie pour le transport d'un flux.}}

\newglossaryentry{wcet}{name=WCET,description={{\it Worst Case Execution Time}, temps maximum
n�cessaire � l'ex�cution d'une tache donn�e}}
\newglossaryentry{switch}{name=switch, description={mat�riel r�seau faisant une
d�capsulation niveau liaison des paquets qu'il re�oit, afin de les rerouter sur la bonne
interface, gr�ce � une capacit� de m�morisation de la position des n{\oe}uds lui �tant connect�s}}

\newglossaryentry{rfc}{name={RFC},description={{\it Request For Comments}}}

\newglossaryentry{mmu}{name=MMU,description={{\it Memory Management Unit}, Unit� mat�rielle int�gr�e au processeur, permettant de fournir une capacit�
d'adressage logique de la m�moire, ainsi qu'un certain nombre de protections en terme de gestion d'acc�s.}}
\newglossaryentry{iommu}{name=I/OMMU,description={{\it In/Out Memory Management Unit}, Unit� mat�rielle int�gr�e au processeur, permettant de contr�ler
les acc�s des �l�ments mat�riel (hors CPU) � la m�moire}}

\newglossaryentry{taskset}{name={Taskset},description={Ensemble de taches. D�finit une liste exhaustive de taches temps r�el, avec un certain nombre de propri�t� comme la p�riode d'instanciation, le WCET ou la {\it deadline}}}
\newglossaryentry{wcep}{name={WCEP}, description={Pour une tache donn�e, d�finit le chemin d'ex�cution le plus long}}
