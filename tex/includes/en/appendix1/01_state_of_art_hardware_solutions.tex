%%
%%
%% state_of_art_hardware_solutions.tex for thesis in /doctorat/these/tex
%%
%% Made by Philippe THIERRY
%% Login   <Philippe THIERRYreseau-libre.net>
%%
%% Started on  Wed Mar 17 13:38:08 2010 Philippe THIERRY
%% Last update Wed Apr  6 10:40:17 2011 Philippe THIERRY
%%

\chapter{State of art of the hardware solution for real-time support}
\doMinitoc

\section{DMA management}

\subsection{Controlling activation period}

\subsection{Impacts}

\section{Cache controllers}

\subsection{Cache locks}

\subsubsection{General principle}

\subsubsection{Static cache locks}

\subsubsection{Dynamic cache locks}

\subsection{Cache partitioning}

\subsection{Les \index{Scratchpad}scratchpads}

\subsubsection{Principe}

\paragraph{}
Scratchpad are local-to-CPU memory, which can be used dynamically by the software in order to host
some of the most frequently executed part of the software elements. Scratchpad are usually
accessible by the CPU core without using the memory bus, reducing by this way the access latency.

\paragraph{}
Depending on the hardware architecture, scratchpad can be used through a allocation of a subpart of
the cache memory, or through an existing local SRAM. Multiple articles (\FIXME{get articles}) have
been written to describes various ways to optimize the scratchpad memory allocation, like code
section prediction and preallocation (\FIXME{get this article}).\\
Figure \ref{fig:cache_scratchpad_simple} describes one of the possible way to allocate a scratchpad.

\begin{figure}
% Generated with LaTeXDraw 2.0.5
% Thu Mar 18 10:31:27 CET 2010
% \usepackage[usenames,dvipsnames]{pstricks}
% \usepackage{epsfig}
% \usepackage{pst-grad} % For gradients
% \usepackage{pst-plot} % For axes
\begin{pdfpic}
\scalebox{1} % Change this value to rescale the drawing.
{
\begin{pspicture}(0,-1.52)(6.5178127,1.52)
\psframe[linewidth=0.04,dimen=outer](6.5178127,1.12)(2.6178124,-0.98)
\psline[linewidth=0.04cm,linestyle=dashed,dash=0.16cm 0.16cm](4.4578123,1.5)(4.4578123,-1.5)
\usefont{T1}{ptm}{m}{it}
\rput(5.9489064,0.85){data}
\usefont{T1}{ptm}{m}{it}
\rput(3.0703125,0.85){code}
\psframe[linewidth=0.04,dimen=outer](4.4778123,-0.26)(2.6178124,-0.98)
\usefont{T1}{ptm}{m}{n}
\rput(3.515,-0.6){\small scratchpad}
\psline[linewidth=0.04,arrowsize=0.05291667cm 2.0,arrowlength=1.4,arrowinset=0.4]{->}(1.8178124,0.54)(1.8178124,-0.1252174)(2.6378126,-0.48)
\usefont{T1}{ptm}{m}{n}
\rput(1.2032813,0.815){\footnotesize r�servation logicielle}
\end{pspicture}
}
\end{pdfpic}

\caption{Reserving a scratchpad in the cache memory}
\label{fig:cache_scratchpad_simple}
\end{figure}

\subsubsection{Impact associated to the scratchpad usage}


\section{SoC and MPSoC}

\subsection{Principle}

\subsection{A word about NUMA architectures}
