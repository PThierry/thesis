%%
%%
%% state_of_art_hardware_solutions.tex for thesis in /doctorat/these/tex
%%
%% Made by Philippe THIERRY
%% Login   <Philippe THIERRYreseau-libre.net>
%%
%% Started on  Wed Mar 17 13:38:08 2010 Philippe THIERRY
%% Last update Tue Jul 20 09:56:12 2010 Philippe THIERRY
%%

\chapter{�tat de l'art des �l�ments mat�riels pour r�pondre au besoin temps r�el}
\doMinitoc

\section{Gestion des DMA}

\subsection{Gestion des p�riodes d'activation}

\subsection{Impacts}

\section{Gestion des caches}

\subsection{Les \index{Scratchpad}scratchpads}

\subsubsection{Principe}

\paragraph{}
La {\it scratchpad} est une capacit� du cache � autoriser une r�servation d'une partie de sa m�moire, anciennement d�di�e
au cache code (cache L1), afin de maintenir une section de code sans risque d'�crasement.\\
Il s'agit, par exemple, de placer une section de code correspondant � l'algorithme d'ordonnancement en cache code, et de
bloquer cette m�moire pour une dur�e ma�tris�e par le logiciel, pouvant aller jusqu'au reset mat�riel.

\FIXME: scratchpad contient .code ou .code,.bss,.rodata,.data... ?
\begin{figure}
% Generated with LaTeXDraw 2.0.5
% Thu Mar 18 10:31:27 CET 2010
% \usepackage[usenames,dvipsnames]{pstricks}
% \usepackage{epsfig}
% \usepackage{pst-grad} % For gradients
% \usepackage{pst-plot} % For axes
\begin{pdfpic}
\scalebox{1} % Change this value to rescale the drawing.
{
\begin{pspicture}(0,-1.52)(6.5178127,1.52)
\psframe[linewidth=0.04,dimen=outer](6.5178127,1.12)(2.6178124,-0.98)
\psline[linewidth=0.04cm,linestyle=dashed,dash=0.16cm 0.16cm](4.4578123,1.5)(4.4578123,-1.5)
\usefont{T1}{ptm}{m}{it}
\rput(5.9489064,0.85){data}
\usefont{T1}{ptm}{m}{it}
\rput(3.0703125,0.85){code}
\psframe[linewidth=0.04,dimen=outer](4.4778123,-0.26)(2.6178124,-0.98)
\usefont{T1}{ptm}{m}{n}
\rput(3.515,-0.6){\small scratchpad}
\psline[linewidth=0.04,arrowsize=0.05291667cm 2.0,arrowlength=1.4,arrowinset=0.4]{->}(1.8178124,0.54)(1.8178124,-0.1252174)(2.6378126,-0.48)
\usefont{T1}{ptm}{m}{n}
\rput(1.2032813,0.815){\footnotesize r�servation logicielle}
\end{pspicture}
}
\end{pdfpic}

\caption{R�servation d'une scratchpad dans le cache}
\label{cache_scratchpad_simple}
\end{figure}

\paragraph{}
L'ex�cution du code mis en scratchpad ne n�cessite plus de cycle de lecture m�moire cot� cache code, et s'ex�cute ainsi
beaucoup plus rapidement. Cependant, l'emploi d'une scratchpad mord sur l'espace m�moire allou� au cache code, et r�duit
en cons�quence la taille du cache code d�di� aux autres applications, provoquant indubitablement un accroissement des
{\it cache-miss}.

\subsubsection{Mise en {\oe}uvre et impact}

\subsubsection{Inconv�nients}

\subsection{Les locks de cache}



\subsubsection{Mise en {\oe}uvre}

\subsubsection{Restriction d'usage et inconv�nients}

\section{Les (MP)SoC}

\subsection{Principe}

\subsection{Int�gration de m�moire � acc�s rapide}
