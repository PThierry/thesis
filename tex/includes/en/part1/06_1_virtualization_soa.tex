%%
%%
%% 06_1_virtualization_soa.tex for  in /home/phil/Travail/These/these/tex
%%
%% Made by Philippe THIERRY
%% Login   <Philippe THIERRYreseau-libre.net>
%%
%% Started on  mar. 13 mars 2012 12:10:50 CET Philippe THIERRY
%% Last update mer. 21 mars 2012 15:35:07 CET Philippe THIERRY
%%

\chapter{State of the art of most known virtualization solutions}

{\it
Virtualization solutions is an complex world of software and hardware solutions, using various
architecture with various efficiency. This chapter proposes a state of art of the most known but
also the most secure virtualization solutions.
}

\doMinitoc

\section{On the general virtualization principles}

\subsection{About virtualization principles}

\paragraph{}
Virtualization is the availability to execute multiple autonomous software environments
concurently on a same hardware. Some virtualization specific component are then defined:
\begin{itemize}
\item An autonomous software component being executed in a partition, with a reduced visibility on
the hardware is a {\it Virtual Machine}.
\item The software and/or hardware subsystem managing the Virtual machines of a given system is a
{\it virtual machine monitor}, as defined in \cite{popek}.
\end{itemize}

\paragraph{}
There is various reasons associated to the usage of virtualization.
Using virtualization may be the consequence of various reasons. Most of the time, it allows to
reduce the hardware cost and the power consumption through a better usage of the system capacities.
Such usage is more and more deployed in modern systems (since the begining of the 21st century).\\
More recently, using virtualization allowing a good flexibility through a dynamic approach of the
virtual machines lifecycle (e.g. Cloud Computing architectures).\\
Using virtualization in order to partition software elements for security has been proposed for
some years. What have been conclude of such security architecture is that virtualization is not a
perfect security solution and should, like native systems, be hardened in order to avoid security
threats. Moreover, some various threats has been demonstrated in some software and more recently
hardware virtualization solutions. These threats are described in \ref{chap:virtsec}.

\paragraph{}
Most of the security breaches of the virtualization solutions is the consequence of some
incompatibility of the hardware with various properties requested by virtualizations solutions.
Usual incompatibilities as been described formaly by Popek and Goldberg in \cite{popek} in 1974.
Today, there is new threats that have been demonstrated on modern systems, based on
\fxnote*{Define covert channel first}{covert channel}.
\fxwarning{foobar}

\subsubsection{About various existing virtualization models}

\paragraph{}
There is various ways to manage virtual machines. Depending on the hardware and software
architecture of the solution, efficiency and security may seriously vary. A virtual machine
monitor supports one (or more) of the following solutions:
\begin{itemize}
\item Host-based virtual machine monitor\\
      Such architecture is based on a complete host operating system. The virtual machine monitor is
      then executed like a process of the host and use admin rights on the host in order to support the
      virtualized software environment needs.
\item Autonomous virtual machine monitor\\
      When the virtual machine monitor is directly executed on the hardware with no host layer,
      it is named {\it hypervisor}. Most of the time, hypervisors are based on a micro-kernel
      design, reducing the overhead of the virtualization to the specific virtual machine management
      routines and basic board support (i.e. interrupt routines initialization blocks)
\end{itemize}

\paragraph{}
Managing virtual machines is a hard task. In \cite{popek}, the authors specify that a virtualized
application execution should have exactly the same behaviour as a native application. Nevertheless,
when executing in user space software that should normally be executed in kernel space, like a
kernel, it try then to execute instructions impacting the system state. Such behaviour may be
difficult to manage in modern architectures.
\begin{itemize}
\item If the instruction is denied in userspace, an exception is rised by the processor core and
the hypervisor can react to it, for example through emulation of the hardware behaviour.
\item If the instruction is allowed in userspace, the hypervisor can't be informed of its execution
and should protect the system against such instruction execution. The hypervisor can prevent such
execution using on line execution flow modifications.
\end{itemize}
Paravirtualization is an easy and efficient way to avoid the usage of sensible instruction in a
virtual machine. It is based on the replacement of any sensible instructions bloc in the virtual
machine applications with a call to a method of the hypervisor. Such method is named {\it
hypercall} and works like a system call in native operating systems. The hypercall execute the
action or any other action having the same result from the virtual machine point of view, in
hypervisor mode. In order to support paravirtualization, the kernel of the virtual machine should
be modified. This can be done for Linux or BSD systems, but not for MS Windows systems.

\subsection{First definition of a virtualizable hardware plateform}

\subsection{Impact on the software stack}

\subsubsection{Formal requirements and properties}

\paragraph{}
About the Popek \& Goldberg formal requirements (1974)

\subsubsection{Incompatibilities of existing hardware}

\subsubsection{On the paravirtualization principles}

\subsubsection{Virtualization friendly processors, their advantages and limitations}

\section{About lightweight and applicative virtualizer}

\subsection{OpenVZ and LXC}

\subsubsection{General architecture}

\subsubsection{Existing threats and security}

\subsection{BSD Jails}

\section{About hypervisor based virtualization}

\subsection{About Xen hypervisor}

\subsubsection{General architecture}

\paragraph{}
Xen is an hypervisor created in the Cambridge UK University in 2003. The Xen hypervisor is Open
Source and is also supported in its commercial version by Citrix.\\
The hypervisor is based on a Linux domain controller. This domain controller is a specific virtual
machine which have administration rights on the hypervisor through specific hypercalls.
The Xen micro-kernel implements board support package for x86 and x86\_64 architecture and has been
ported to ARM architecture. Event if the hypervisor is a micro-kernel, its source code is
relatively big (approximatively 250 Klines).\\
The hypervisor support in the same time paravirtualisation, through the implementation of a
complete set of hypercalls, and Hardware-based virtualization for Intel VT and AMD-V.
The hypervisor scheduling policy is based on a non Real-Time scheduler, but also supports a S-EDF
scheduling policy definining per VM minimum granted time slot in a given time period.
\paragraph{}
The usage of a domain manager based on a complete Linux system permits an easy management of the
system but generates a certain overhead and reduce the security of the overall system.
Xen guarantee a complete spacial (memory) partitioning of the VM, avoiding classical memory based
covert channel. Nevertheless, Xen does not support a strict temporal partitioning of the virtual machine.

%\subsubsection{Existing threats and security}

\paragraph{The QubesOs security solution}

\subsection{VMWare ESX solution}

\subsubsection{On VMWare solution architecture}

\subsubsection{Existing threats and security}

\subsection{Polyxene security solution}

\subsection{Sysgo PikeOs separation kernel}

\subsubsection{General architecture}

\subsubsection{Threats and security}

\subsection{About WindRiver \& GreenHills solutions}

\section{conclusion about the virtualization technologies}
