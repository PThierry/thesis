%%
%%
%% scheduling_policies.tex for thesis in /doctorat/these/tex
%%
%% Made by Philippe THIERRY
%% Login   <Philippe THIERRYreseau-libre.net>
%%
%% Started on  Fri Mar 12 16:28:03 2010 Philippe THIERRY
%% Last update Mon Apr  4 16:03:24 2011 Philippe THIERRY
%%

\chapter{Real-time schedulers familly}

{\it
In order to schedule Real-Time tasks, multiple specific schedulers have been implemented in order
to propose various properties. Choosing a Real-Time scheduler may depend on various needs, like
efficient energy management or security considerations.
}

\doMinitoc

\paragraph{}
Various Real-Time schedulers possess several properties impacting the tasks management and the
hardware architecture support like SMP. This chapter explains each property separately. Considering
the set of all scheduling property, each scheduler is considered like a specific subset, supporting
some of all these properties.

\section{Classification of dynamic schedulers}

\subsection{On-line scheduling}

\paragraph{}
When a scheduling policy is done on-line, it is then defined depending on the current eligible
tasks list. The scheduler is then able to react to unpredicted variations on the system load, like
a job processor's overconsumption.

\subsection{Out-of-line scheduling}

\paragraph{}
An out-of-line scheduling is defined before the effective system execution. Whatever happend, the
system guarantee that the scheduling of each task is done respecting a pre-defined time sharing.\\
Such policy is then predictable, but may generates deadline misses on unespected slows down. This
is the case, for example, of TDMA (Time-Division Multiple Access) scheduling policies.

\section{Classification based on scheduling schemes}

\paragraph{}
Figure \ref{fig:sched_automaton} describes the generic state automaton of any task scheduling
policy. Such automaton is used in every scheduling policies.
\begin{figure}[h]
% Generated with LaTeXDraw 2.0.5
% Sun Feb 14 15:38:44 CET 2010
% \usepackage[usenames,dvipsnames]{pstricks}
% \usepackage{epsfig}
% \usepackage{pst-grad} % For gradients
% \usepackage{pst-plot} % For axes
\begin{pdfpic}
\scalebox{0.8} % Change this value to rescale the drawing.
{
\begin{pspicture}(0,-3.13)(6.7,3.13)
\pscircle[linewidth=0.04,dimen=outer](3.36,2.17){0.96}
\pscircle[linewidth=0.04,dimen=outer](5.74,-0.03){0.96}
\pscircle[linewidth=0.04,dimen=outer](3.34,-2.17){0.96}
\pscircle[linewidth=0.04,dimen=outer](0.96,-0.03){0.96}
\usefont{T1}{pcr}{m}{n}
\rput(0.9532812,0.015){Running}
\usefont{T1}{pcr}{m}{n}
\rput(3.359375,2.175){Preempt}
\usefont{T1}{pcr}{m}{n}
\rput(5.7178125,-0.0050){Waiting}
\usefont{T1}{pcr}{m}{n}
\rput(3.335625,-2.145){Ready}
\psline[linewidth=0.03cm,arrowsize=0.1cm 2.0,arrowlength=1.4,arrowinset=0.4]{->}(1.38,0.81)(2.48,1.77)
\psline[linewidth=0.04cm,arrowsize=0.1cm 2.0,arrowlength=1.4,arrowinset=0.4]{->}(4.08,1.59)(5.04,0.59)
\psline[linewidth=0.04cm,arrowsize=0.1cm 2.0,arrowlength=1.4,arrowinset=0.4]{->}(5.1,-0.73)(4.12,-1.57)
\psline[linewidth=0.04cm,arrowsize=0.1cm 2.0,arrowlength=1.4,arrowinset=0.4]{<-}(2.66,-1.53)(1.7,-0.57)
\psline[linewidth=0.04cm,arrowsize=0.1cm 2.0,arrowlength=1.4,arrowinset=0.4]{->}(2.64,1.55)(1.62,0.65)
\psline[linewidth=0.04cm,arrowsize=0.1cm 2.0,arrowlength=1.4,arrowinset=0.4]{<-}(1.54,-0.79)(2.48,-1.73)
\end{pspicture}
}
\end{pdfpic}

\caption{Dynamic scheduling generic automaton\label{fig:sched_automaton}}
\end{figure}

\subsection{Preemptive schedulers}

\paragraph{}
Preemptive schedulers preempt the currently executed job after a specific duration, named {\it
time \index{quantum}quantum}. As a consequence, a job may be pushed back from the processor
multiple time before it finishes. The scheduling overhead is then increased, but the system
reactivity is also increased.

\paragraph{}
Usually, preemptive schedulers are time-driven. Their execution is done with a specific period,
depending on the system configuration. It is the case of most of the Best-Effort non-Real-Time
schedulers and of some of Real-Time schedulers like Round-Robin.\\
Some Real-Time preemptive scheduler are event-driven. Their execution is due to a specific event,
like a new job arrival. This is the case of EDF, which preempts the current job only if a more
prioritized job just arrives.


\subsection{Non-preemptive schedulers}

\paragraph{}
Non-preemptive schedulers guarantee that the execution of a given job is done without preemption
due to the scheduling policy. First-In First-Out is a Real-Time non-preemptive scheduler.

\subsection{Mixed scheduler}

\paragraph{}
Some schedulers have been implemented in order to support in the same time preemptive and
non-preemptive policies. Segregation can be done using for example the task priority, considering
high priority tasks as non-preemptive and low priority ones as preemptive tasks.

\section{Classification based on idleness}

\subsection{Idle schedulers}

Idle schedulers are preemptive schedulers. They define fixed time quantum for each job
sub-execution, and do not load a job by advance if the previous job didn't consume all its
quantum.\\
Such property maintain the predictability of the scheduling policy at the loss of effectiveness.

\subsection{Non-idle schedulers}

Contrarily to Idle schedulers, such schedulers elect by advance a job if the previous one didn't
consume all its quantum. In the same manner, such schedulers are preemptive.

\section{Classification based on priority policy}

\subsection{Fixed priority schedulers}

\paragraph{}
Each task possess a fixed priority, defined during its creation. There is no task aging.

\subsection{Dynamic priority schedulers}

\paragraph{}
Each task possess two priorities. One defined during its creation and used at each new job arrival,
and one due to the current job aging. This is the case of EDF, which check the current relative
deadline in order to elect the job having the earliest one.\\
Priority variation can be done through multiple criteria like aging, previous CPU consumption or
any other variable associated to a given job.

\subsection{Hybrid priority schedulers}

\paragraph{}
In the same manner as mixed schedulers, hybrid priority schedulers support mixed dynamic and static
priorities, depending on a segregative property. Such schedulers are denoted {\it FP/DP} schedulers
(Fixed Priority/Dynamic Priority).

\section{About SMP-aware schedulers}

\paragraph{}
Nowadays, most of hardware architectures are based on SMP systems, using in the same time multiple
processors in order to accelerate the system capacities and reducing the overall consumption
compared with Multi-processors architecture.

\paragraph{}
There are three main families of SMP-aware Real-Time scheduling policies:\\
\begin{enumerate}
\item \index{Global scheduling}Global scheduling
\item \index{Partitioned scheduling}Partitioned scheduling
\item \index{Semi-partitioned scheduling}Semi-partitioned scheduling
\end{enumerate}

\subsection{Global scheduling}

\paragraph{}
Whatever the number of core is, there is one eligible job queue. When the scheduler is executed on
a given core (trough time-driven execution for example) the elected job is load in the core.\\
Such scheduling policy have various impact on the overall schedulability, like cache-based
preemption delay due to job movement between cores.

\subsection{Partitioned scheduling}

\paragraph{}
Using a partitionned scheduling, we define one scheduling queue per core or cores pool. In
this pool, a global scheduling policy is implemented. If the pool contains one and only one core,
its execution policy is the same as in mono-core architectures, otherwise, global scheduling is
used.

\subsection{Semi-partitioned scheduling}

\paragraph{}
Semi-partitioned scheduling support controlled job movement through multiple cores. There is two
types of semi-partitioned scheduling policies:
\begin{itemize}
\item job partitioning
\item instance partitioning
\end{itemize}

\subsubsection{Job partitioning}

\paragraph{}
Using job paritioning, the scheduler associates one job with one and only one core. Another
instantiation of the given task may be executed in another core. Thus the choice of the newly
selected core is predefined in an off-line moving policy.

\subsubsection{Instance partitioning}

\paragraph{}
Using instance partitioning, the scheduler allows movement of a given job between cores. Thus, such
movements are controlled through a policy defining when and where a job is preempted in order to be
moved in another core .\\
As describes in \FIXME{article: Kato} a currently executed job can migrate from a core to another
using the following algorithm \FIXME{mkproper}:\\
\begin{enumerate}
\item we denote $s$ the number of active cores.
\item Considering a task $\tau_{i}$, when an associated job is executed during a time quantum $Q$ of a
duration $L = \frac{D_i}{s}$, then the job is moved to the first free core.
\end{enumerate}

\paragraph{}
The goal is to reduce the number of used core, increasing their load. Thus, such algorithm does
not support unexpected variations of the WCET, which has a great impact on the schedulability of
the overall taskset.

\subsection{Sustainable and unsustainable schedulers}

\paragraph{}
A sustainable scheduler support elasticity of the Real-Time properties, like the WCET. On the
contrary, unsustainable scheduler doesn't guarantee the system's schedulability if any of the
Real-Time properties is not respected.

\section{A word about hierarchical scheduling}

\paragraph{}
Hierarchical schedulers are defined like multi-level scheduling policies. Such architecture allows
concurrent execution of task sets with different scheduling policies, managed by an underlying
scheduling policy.

\subsection{Defining a hierarchy}

\paragraph{}
When defining a hierarchical scheduling policy, each task should be segregated in order to define by
which subset it is scheduled. In the same time, subsets should be prioritized in order to elect a
job between each subset elected one.

\subsection{Multi-schedulers priority management}

\paragraph{}
A lot of work has been done on hierarchical scheduling.\FIXME{add articles, cf. hypervisor\_sched}

\subsection{Toward virtualization-based hierarchical scheduling}

\paragraph{}
In hyervisor-based architectures, hierarchical scheduling is used in order to schedule various
virtual machines or virtual environments. Depending on the need, the hypervisor scheduler can be a
classical on-line scheduler like EDF or a strict time scharing scheduler like TDM ones.\\
In the case of virtual machine scheduling, the unerlying hypervisor scheduler may not be informed
of the virtual machine scheduler's needs.
