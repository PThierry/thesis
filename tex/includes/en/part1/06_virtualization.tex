%%
%%
%% 06_virtualization.tex for  in /home/phil/Travail/These/these/tex
%%
%% Made by Philippe THIERRY
%% Login   <Philippe THIERRYreseau-libre.net>
%%
%% Started on  mer. 08 févr. 2012 14:24:59 CET Philippe THIERRY
%% Last update mer. 08 févr. 2012 14:36:17 CET Philippe THIERRY
%%

\chapter{About the security through virtualization}

{\it
Virtualization is one of the existing solution for security partitioning. However, as
virtualization solutions have efficient security properties as they generate some new threats and
should be considered carefully.\\
This chapter describes the existing virtualization solutions and technologies, then targets the security enhanced
hypervisors and their impact on the system security and schedulability.
}

\doMinitoc

\section{On the virtualization principles}

\subsection{About virtualization principles}

\subsubsection{About various existing virtualization models}

\paragraph{}
Level 1, level 2, application based, Para, HVM, FullVM and definition of names

\subsection{First definition of a virtualizable hardware plateform}

\subsection{Impact on the software stack}

\subsubsection{Formal requirements and properties}

\paragraph{}
About the popek \& Goldberg formal requirements (1972)

\subsubsection{Incompatibilities of existing hardware}

\subsubsection{On the paravirtualization principles}

\subsubsection{Virtualization friendly processors, their advantages and limitations}

\section{State of art of the L1 \& L2 virtualization solutions}

\subsection{About Xen and hybrid micro-kernel based virtualizers}

\subsubsection{LXC, jail and mono-kernel partitioning solutions}

\subsubsection{About security and safety enhanced hypervisors}

\paragraph{Sysgo PikeOS}

\paragraph{WindRiver Integrity}

\paragraph{GreenHills solution}


