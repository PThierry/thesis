%%
%%
%% 01_scope_definition.tex for thesis in /doctorat/these/tex
%%
%% Made by Philippe THIERRY
%% Login   <Philippe THIERRYreseau-libre.net>
%%
%% Started on  Thu Jul 15 17:46:14 2010 Philippe THIERRY
%% Last update mer. 13 juil. 2011 11:24:07 CEST Philippe THIERRY
%%

\chapter{About the security study scope}

{\it
IT security is a large domain. In this thesis, the focus is done on the overall executing
environment security.\\
Are taken into account elements like the memory management, process execution and roles definition,
time and space partitioning.
}

\doMinitoc

\section{Introduction}

\subsection{Defining the security scope}

\paragraph{}
In this document is only addressed, under the term of {\it security}, the {\it security of
Information \& Technology} part. Safeness is considered through the existing Real-Time software
architecture and associated supported hardware.

\subsection{Description of security formalisms}

\paragraph{}
In {\it security of Informations and Technologies}, a complete formalism has been defined since
1973 by Bell \& LaPadula. This formalism uses some terms and principles in software security, as
described in Table \ref{tab:secformalfirst}.

\begin{table}
\begin{tabular}{|p{5cm}|p{10cm}|}
\hline
  Formalism & Signification \\
\hline
\hline
  Subjects & \begin{minipage}{10cm}Any program which can be executed on the system\end{minipage} \\
\hline
  Objects & \begin{minipage}{10cm}Any software element of the system. They can be data, file(s), or
  subjects\end{minipage} \\
\hline
  The need-to-know & \begin{minipage}{10cm}The knowledge of a given Subject in term of Objects
  visibility or access\end{minipage} \\
\hline
  Transaction & \begin{minipage}{10cm}Any communication between a Subject and an Object. These
  transactions are controlled depending on the system policy\end{minipage} \\
\hline
\end{tabular}
\caption{Most important formal term in software security\label{tab:secformalfirst}}
\end{table}

\paragraph{}
To this basic elements, three main security guarantees are defined in Table \ref{tab:secguarantee}.

\begin{table}
\begin{tabular}{|p{5cm}|p{10cm}|}
\hline
  Guarantee & Signification \\
\hline
\hline
  Confidentiality & \begin{minipage}{10cm}The guarantee that a given object is can't be accessed by
  an unauthorized Subject\end{minipage} \\
\hline
  Integrity & \begin{minipage}{10cm}The guarantee that a given object can't be modified by an
  unauthorized Subject\end{minipage} \\
\hline
  Non-repudiation & \begin{minipage}{10cm}The guarantee that a Subject can't challenge the proof of
  its own unauthorized action\end{minipage} \\
\hline
\end{tabular}
\caption{Security guarantees\label{tab:secguarantee}}
\end{table}



\paragraph{}
Are considered here security elements associated to:
\begin{itemize}
\item The operating system implementation
\item The memory management and threats
\item The hardware access
\item The hardware-based side-channels and covert-channels threats
\end{itemize}

\section{Defining associated security functions}
\label{sec:cc_fdp}

\paragraph{}
In this document are considered the security function of the Common Criteria FDP
class\footnote{bibliography:\FIXME: CCPART2V3.1R3.pdf}. This class defines the protection function
of the user data. These functions are separated in four parts:
\begin{enumerate}
\item User data protection policy
\item User data protection architecture
\item User data off-line backup, import and export
\item Security if the user data communication between trusted components. This last part is not
considered in this document.
\end{enumerate}

\section{Defining the security study vector}

\paragraph{}
The system security study is done through a bottom-up approach, taking into account hardware
security problematic and architecture, then the kernel and operating system security functions and
implementation, to finish with the user space software protection and partitioning.\\
All these elements are considered in order to be compliant to Real-Time problematics in order to
maintain both Real-Time and security considerations.
