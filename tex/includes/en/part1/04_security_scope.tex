%%
%%
%% 01_scope_definition.tex for thesis in /doctorat/these/tex
%%
%% Made by Philippe THIERRY
%% Login   <Philippe THIERRYreseau-libre.net>
%%
%% Started on  Thu Jul 15 17:46:14 2010 Philippe THIERRY
%% Last update Mon Apr  4 17:16:16 2011 Philippe THIERRY
%%

\chapter{About the security study scope}

{\it
IT security is a large domain. In this thesis, the focus is done on the overall executing
environment security.\\
Are taken into account elements like the memory management, process execution and roles definition,
time and space partitioning.
}

\doMinitoc

\section{Introduction}

\paragraph{}
In this document is only addressed, under the term of {\it security}, the {\it security of
Information \& Technology} part. Safeness is considered through the existing Real-Time software
architecture and associated supported hardware.

\paragraph{}
Security of Information \& Technology is a large domain. In this document are only considered
security threats and fragility associated to the considered software/hardware couple. Network
security, including all associated software and hardware is considered out of scope.

\paragraph{}
Are considered here security elements associated to:
\begin{itemize}
\item The operating system implementation
\item The memory management and threats
\item The hardware access
\item The hardware-based side-channels and covert-channels threats
\end{itemize}

\section{Defining associated security functions}
\label{sec:cc_fdp}

\paragraph{}
In this document are considered the security function of the Common Criteria FDP
class\footnote{bibliography:\FIXME: CCPART2V3.1R3.pdf}. This class defines the protection function
of the user data. These functions are separated in four parts:
\begin{enumerate}
\item User data protection policy
\item User data protection architecture
\item User data off-line backup, import and export
\item Security if the user data communication between trusted components. This last part is not
considered in this document.
\end{enumerate}

\section{Defining the security study vector}

\paragraph{}
The system security study is done through a bottom-up approach, taking into account hardware
security porblematic and architecture, then the kernel and operating system security functions and
implementation, to finish with the user space software protection and partitioning.\\
All these elements are considered in order to be compliant to Real-Time problematics in order to
maintain both Real-Time and security considerations.
