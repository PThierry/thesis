%%
%%
%% software_impacts.tex for thesis in /doctorat/these/tex
%%
%% Made by Philippe THIERRY
%% Login   <Philippe THIERRYreseau-libre.net>
%%
%% Started on  Fri Mar 12 16:42:22 2010 Philippe THIERRY
%% Last update Mon Apr  4 17:49:31 2011 Philippe THIERRY
%%

\chapter{Software impact}
\doMinitoc

\section{On the operating system complexity}

%\subsubsection{}

\section{The scheduler overhead}

\paragraph{}
L'ex�cution de l'ordonnanceur poss�de �galement un co�t.  Selon l'impl�mentation, ce co�t peut  �tre
en $O(1)$, en $O(n)$ voire plus.  Le co�t  d'ex�cution  de  l'ordonnanceur  est  bornable  et  l'on
pourrait int�grer son co�t comme une tache.  Cependant, son sh�ma d'ex�cution ne  correspond  pas  �
celui de l'ensemble de tache qu'il ordonnance.  Ainsi ce dernier ne peut �tre  consid�r�  comme  une
tache de l'ensemble de taches courant, mais comme un surco�t � int�grer � l'ensemble des  taches  de
l'ensemble de tache.\\
Il est donc n�cessaire � la fois de conna�tre le WCET de l'ordonnanceur, mais �galement de conna�tre
le nombre de pr�emption maximum de chaque tache afin de pouvoir d�finir une borne maximum du surco�t
de l'ordonnanceur.  La difficult� revient ici � d�finir une  borne  la  moins  pessimiste  possible.

\section{Software architecture}

\subsection{On the software call tree and branching impact}

\subsection{The memory consumption and memory mapping}

Data alignment should also be in this section.

\subsection{The I/O usage and its impact}

\subsection{Critical sections and lock management}

\subsection{Software parallelism}

Threads, parall�lisations par optimisation

\section{Toolchain behaviour and effects}

Alignement  des   structures   de   donn�es   sur   les   lignes   de   caches,   en   fonction   de
l'algorithme de gestion de cache.
Optimisation...
