%%
%%
%% 01_definition_modele.tex for thesis in /doctorat/these/tex
%%
%% Made by Philippe THIERRY
%% Login   <Philippe THIERRYreseau-libre.net>
%%
%% Started on  Tue Aug 31 13:12:11 2010 Philippe THIERRY
%% Last update Tue Jul 12 17:05:44 2011 Philippe THIERRY
%%

\chapter{On the definition of a pure Real-Time model}

\section{About the target of this model}

\paragraph{}
In this chapter, we define a real-time model for multiple soft real-time sporadic task-sets on
hypervisor based architectures. Targeted tasks are critical network flow forwarding tasks with
deadline support.

\chapter{On the definition of a MILS Real-Time aware security model}

\section{On the need of the security model formalism}

\paragraph{}
As defined in various document like \cite{formalsec}, security formalism for modern computer
architecture is a necessary step in the definition of a efficient security-aware system.\\
Such formalism permits to define properly the meaning of the term {\it security}. Through this
definition, it permits to define the scope of the security elements that should be integrated in
the system, and consequently to the security requirements and function specific to that system.

\paragraph{}
In this thesis, the goal is to define security elements that should be integrated in the
architecture in order to respond to the need of a Multi Independent Level of Security architecture
with real-time constraints for critical network flows management.

\paragraph{}
Through a formalized security model, specific security and safety requirement are derived and
integrated in the overall system architecture.

\section{Formal definitions}

\paragraph{}
The model which is defined in this chapter is based on \cite{secmathfund}. This model denotes
various formal elements in order to define a complete security model for modern computer system.
The list of formal elements defined in \cite{secmathfund} is re-written in Table
\ref{tab:secmathfund} for a better comprehension.\\
In this thesis, this model is updated in order to take into account Real-Time requirement.

\paragraph{}
In \ref{tab:secmathfund}, L. LaPadula proposes a formal definition for various entities and
security elements in order to define his security model. In this thesis, we defines some new
in order to take into account real-time needs but also hardware impacts on the system security.

\begin{table}[ht]
\begin{tabular}{|p{3cm}|p{6cm}|p{7cm}|}
\hline
Set & Elements & Semantics \\
\hline
\hline
S & $\{S_1, S_2, \hdots, S_n\}$ & {\it Subjects}; processes, programs in execution \\
\hline
O & $\{O_1, O_2, \hdots, O_m\}$ & {\it Objects}; data, files, programs, subjects \\
\hline
C & \begin{minipage}{5.5cm}$\{C_1, C_2, \hdots, C_q\}$\\$\{C_1 > C_2 > \hdots > C_q\} $\end{minipage} & {\it
Classifications}; clearance level of a subject, classification of an object \\
\hline
K & $\{K_1, K_2, \hdots, K_r\}$ & {\it need-to-know categories}; project numbers, access privileges \\
\hline
A & $\{A_1, A_2, \hdots, A_p\}$ & {\it Access attributes}; read, write, copy, append, owner, control \\
\hline
R & $\{R_1, R_2, \hdots, R_u\}$ & {\it requests}; input commands, requests for access to objects by
subjects \\
\hline
D & $\{D_1, D_2, \hdots, D_v\}$ & {\it Decisions}; outputs, answers,  "yes", "no", "error" \\
\hline
P$_\alpha$ & all subsets of $\alpha$ & {\it power set of $\alpha$} \\
\hline
$\alpha^\beta$ & all functions from the set $\beta$ to the set $\alpha$ & { } \\
\hline
$\alpha  \times \beta$ & $\{a,b): a \in \alpha, b \in \beta\}$ & {\it Cartesian product of the sets
$\alpha$ and $\beta$} \\
\hline
F & \begin{minipage}{5.5cm}$C \times C^O \times (PK)^S \times (PK)^O$\\ an arbitrary element of F is written
$f=(f_1,f_2,f_3,f_4)$\end{minipage} & \begin{minipage}{6.5cm}{\it classification/need-to-know
vectors};\\$f_1$: subject-classification function\\
           $f_2$: object-classification function\\
           $f_3$: subject-need-to-know function\\
           $f_4$: object-need-to-know function
           \end{minipage} \\
\hline
X & \begin{minipage}{5.5cm}$R^T$\\ an arbitrary element of X is written x\end{minipage} & {\it request sequences} \\
\hline
Y & \begin{minipage}{5.5cm}$D^T$\\ an arbitrary element of Y is written y\end{minipage} & {\it request sequences} \\
\hline
M & \begin{minipage}{5.5cm}$\{M_1, M_2, \hdots, M_c\}$,\\ $c=nm2^p$;\\an element $M_k$ of $M$ is an $n
\times m$ matrix with entries from PA; the (i,j)-entry of $M_k$ shows $S_i$'s access attributes
relative to $O_j$\end{minipage} & {\it access matrices} \\
\hline
V & $P(S \times O) \times M \times F$ & {\it states} \\
\hline
Z & \begin{minipage}{5.5cm}$V^T$\\ an arbitrary element of Z is written z; $z_t \in z$ is the t-th
stae in the state sequence z\end{minipage} & {\it state sequences} \\
\hline
\end{tabular}
\caption{List of formalized element defined by LaPadula for a mathematical security model\label{tab:secmathfund}}
\end{table}

In order to take into account real-time and hypervisor based architectures, we define new elements
in Table \ref{tab:secmathfundmils}

\begin{table}[ht]
\begin{tabular}{|p{3cm}|p{6cm}|p{7cm}|}
\hline
Set & Elements & Semantics \\
\hline
\hline
$\mathbb{V}$ & ${\mathbb{V}_1, \mathbb{V}_2, \hdots, \mathbb{V}_z}$ & {\it Container}; autonomous partitions containing a given set of objects \\
\hline
$\mathbb{R}$ & $\{\mathbb{R}_1, \mathbb{R}_2, \hdots, \mathbb{R}_t\}$ & {\it RT requirements}; set
of Real-Time specific requirements \\
\hline
\end{tabular}
\caption{Real-Time virtualization-aware new elements for separation-kernel based real-time MILS
architecture\label{tab:secmathfundmils}}
\end{table}

\paragraph{}
We define some Assertions on the various set inclusion.
\begin{assertion}
$\forall (i, j), i \in \{1, \hdots, z \}, j \in \{ 1, \hdots, z \}, i \neq j, \mathbb{V}_i \bigcap \mathbb{V}_j = \emptyset $
\end{assertion}
\begin{assertion}
$\forall o \in O, \exists i \in \{ 1, \hdots, z \}, o \in \mathbb{V}_i$
\end{assertion}


