%%
%%
%% remerciements.tex for  in /home/phil/Travail/Doctorat/these/tex
%%
%% Made by phil
%% Login   <philreseau-libre.net>
%%
%% Started on  dim. 13 avril 2014 15:28:36 CEST phil
%% Last update ven. 18 avril 2014 11:35:19 CEST phil
%%

\chapter*{Remerciements}

\par
On m'avais dit que le diplôme d'ingénieur, ce n'est pas suffisant pour pouvoir
{\it tenir} une carrière entière. On me disait "tu devrais faire un MBA", car
oui, avoir un diplôme de finances, allez savoir pourquoi, il paraît que ça
sert dans l'ingénierie... Mais voila, la finance, je déteste ça... Même à la
maison, c'est ma femme qui gère les comptes (surement beaucoup mieux que
moi).\\
Mais au final, j'ai toujours préféré la technique. Même si dans l'industrie, elle
n'est à mon sens pas considérée à sa juste valeur. J'ai donc voulu faire un
doctorat. N'allez pas penser que j'ai fait un doctorat juste pour le plaisir
d'avoir un diplôme en plus. Si ça avait été le cas, j'aurais fini par
m'échapper de se long et obscur tunnel qu'est une thèse, bien avant sa fin.
J'ai choisi un doctorat car cela avait un sens pour moi: me spécialiser dans un domaine
technique qui m'intéresse, et m'enrichir auprès de gens qui ont su m'apprendre
beaucoup de choses durant ces dernières années. Il est maintenant temps de
tous les remercier.

\par
Il n'y a pas d'ordre particulier à mes remerciements, il ne faut donc pas en
chercher un. Néanmoins, je ne peux pas ne pas commencer ceux-ci par les deux
personnes qui m'ont accompagné pendant ces (presque) cinq ans, m'ont enrichi
de leur compétences et de leur connaissances, ainsi que de leur carnet
d'adresses.\\
Merci donc à {\sc Jean-Marc Lacroix}, mon responsable scientifique, qui a su
me faire faire autre chose qu'un doctorat pendant toutes ces années, et qui a
su me faire participer à diverses affaires, appels d'offre, revues
d'architecture et j'en passe.\\
Merci également à {\sc Laurent George}, mon directeur de thèse. Merci d'avoir
accepté de m'encadrer pendant cette thèse, d'avoir pris du temps pour me
former dans ce domaine complexe qu'est le temps réel, et d'avoir accepter ma
tendance maladive à vouloir mettre des problématiques de sécurité partout.
Merci de m'avoir emmener à toutes ces conférences, de m'avoir appris à être
rigoureux et synthétique, dans tous ces articles et pour finir dans ce présent
document.

\par
Par respect pour eux, je me dois de remercier les membre de mon jury.

\par
Une thèse, c'est beaucoup de solitude dans son travail. Je souhaite donc
remercier tous les camarades qui ont permis de bonnes tranches de rigolades,
tout au long de ces années. Merci donc, dans le désordre le plus total, à {\sc
Pierre Courbin}, {\sc Sylvain Leroy}, {\sc Clément Duart} ou encore {\sc Zig} et {\sc
Zog} (à savoir Bertrand Fuganiolli et Vincent Huitric).

\par
Je tiens à remercier tout particulièrement {\sc Jean-François Hermant}, qui a
beaucoup participé à mes travaux et qui m'a beaucoup aidé dans les premiers
articles que j'ai publié. Merci aussi à Laurent pour sa patience dans le cadre
de la relecture de ma thèse, malgré mes {\it TOCs} d'usage de certains mots ou
expressions, dont il a ordonné la chasse durant plusieurs mois ! 

\par
Bien entendu, je ne peux pas oublier {\sc Fabien Germain}, qui a été fort de
beaucoup de conseils durant toutes ces années, tant sur le fond que sur la
forme, principalement sur tous les travaux que j'ai effectué autour du domaine
de la sécurité.
